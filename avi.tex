\documentclass[11pt,preprint]{aastex}

\begin{document}

\section{Trajectory in a homogeneous Universe}

We consider the trajectory of a fast non-relativistic star, that
reached the vicinity of the Milky-Way galaxy at the present time $t_0$
with a measured 3D speed $v_0$. It is well-known that the peculiar
momentum of an object relative to the Hubble flow, declines inversely
with the scale-factor $a(t)=(1+z)^{-1}$ in an expanding
universe\footnote{Naturally, the de Broglie wavelength of a particle
(which is inversely proportional to its momentum) is stretched by the
scale factor $a(t)$ like any other physical lengthscale in the
expanding Universe.}. This implies that the peculiar velocity of the
star at earlier times $t<t_0$ was,
\begin{equation}
v=v_0/a(t).
\label{eq:1}
\end{equation}

To leading order, we assume a flat homogeneous and isotropic Universe
described by the metric, $ds^2=c^2dt^2-a^2(t)(dr^2+r^2d\Omega)$, where
$r$ is the comoving (present-day) radius relative to the observer. For
simplicity, we consider the case where the source galaxy from where
the star was ejected at time $t_{ej}$ is much farther away ($\sim {\rm
Gpc}$) than the distance of the star from the observer ($\sim {\rm
Mpc}$). In this regime, the trajectory of the star is nearly radial
towards the Milky Way, with a time-dependent velocity
$v(t)=a(t)dr/dt$. By substituting Eq. (\ref{eq:1}), we get $dr=v_0
dt/a^2(t)$, and after integrating both sides of this equation we find,
\begin{equation}
r(t_{ej})=v_0 \int_{t_{ej}}^{t_0} {dt \over a^2(t)} .
\label{eq:2}
\end{equation}

If the star was ejected when it was much younger than its present age,
$t_\star$, then one could infer $t_{ej}=t_0-t_\star$ from a
spectroscopic measurement of $t_\star$.

After identifying the source galaxy from the flight time of the star
$t_\star$ and its spectroscopically-measured radial velocity $v_0$,
one can observe the spectrum of the host galaxy and determine its
redshift $z$.  The cosmological evolution of the scale factor,
$a(t)=(1+z)^{-1}$, depends on cosmological parameters through the
cosmological Friedmann equation. In principle, if the measurement
accuracy is sufficiently high, one could constrain cosmological
parameters (such as the equation of state of dark matter) by requiring
that the source redshift as measured through photons at $t_0$ would
agree with the comoving distance traveled by a star with a present-day
velocity $v_0$ over its lifetime $t_\star$
(cf. Eq. \ref{eq:2}). Interestingly, the photons were emitted after
the star left the galaxy, so that they would arrive to us at the same
time.  This provides a new variant of the conventional ``Hubble
Diagram'' which traditionally uses only photons. The proposed
measurementwould therefore constitute a unique test of the equivalence
principle in General Relativity and of the standard model of
cosmology.

If the source galaxy is far away, the star and galaxy may almost
overlap on the sky. The proper motion of the star can be used to
pinpoint the parent galaxy more accurately. In case of an overlap, one
would need to disentangle the spectral lines of the star from those of
its parent galaxy.

\section{Deflection by a mass concentration}

A non-relativistic star passing within a distance of closest approach
(impact parameter) $b$ from a spherical mass concentration $M(r)$
would be deflected by an angle (see J. Binney \& S. Tremaine, Galactic
Dynamics (1987), page 189),
\begin{equation}
\theta={2GM(b)\over v^2 b}={2V_c^2\over
v^2}=41.25^{\prime\prime}\left[{(V_c/300~{\rm km~s^{-1}})\over
(v/0.1c)}\right]^2 ,
\label{eq:3}
\end{equation}
where $V_c=\sqrt{GM(b)/b}$ is the circular velocity of the deflector.
Note that aside from the substitution of $v$ for $c$, there is a
factor of 2 missing in the deflection angle of a non-relativistic
particle relative to that of a photon (which couples equally to the
time and space components of the metric).

The probability for deflection by the core of an intervening galaxy or
a cluster of galaxies, is small $\lesssim 1\%$ and deflection by
large-scale structures (the equivalent of weak lensing for photons) is
likely to dominate for most of the ultrafast stars. The characteristic
deflection angle as a function of distance can be inferred from the
value calculated for photons (Refregier, A., ARA\&A, 41, 645, 2003)
times the factor $c^2/2v^2$. For $v\sim 0.1c$, large scale structure
is expected to introduce a characteristic deflection of several
degrees for a source galaxy at a cosmological distance $\gtrsim {\rm
Gpc}$.

Probability is of order unit at the virial radius, the calcualtion above only considers the core. Multiple deflections need to be considered

\end{document}